Metric trees are an efficient way to search for object based on some metric.
We will give a quicker overview of our implementation of a VP-Tree\footnote{VP-Tree stands for vantage point tree} \cite{j2kun_2014},
a particular favor of metric tree that is convient because of its not dependence on the dimensionality of a vector space.
This is useful because it allows for metric spaces with no dimension such as that of type signatures.

The goal of a metric tree is to take some point $p$, and some radius, and return all the points that are within the radius away from $p$.
The point $p$ and the radius can together be thought of as a ball, and the goal of a metric tree is to return all the points in the metric space 
    that reside in that ball.
The idea of the VP-Tree is split the metric space recursively into interiors and exteriors.
An interior is a metric ball, that is a point in the space with some radius.
An exterior is just the complement of some interior.
By splitting a space up this way, 
we can save ourselves a substatinal amount of effort by only search the interior if the target ball intersects the interior,
and only searching the exteriors if the target ball extends beyond an interiors.
A VP-Tree is defined as a reursive data type parameterized over the type of the metric, denoted $X$ here:

\begin{align*}
\text{VP-Tree } X &= \{ \\ 
interior &: \text{VP-Tree } X, \\ 
exterior &: \text{VP-Tree } X, \\ 
self &: X, \\ 
radius &: \Rnn \\
\}
\end{align*}

In this definition we note a $\text{VP-Tree } X$ can be $\null$

To aid in understanding it is worth noting the similarity between a VP-Tree and normal binary search tree.
In a normal binary search tree,
if we were to do a range query,
we explore the left child of a node if the node's value is within our range,
and same with the right child.
As discussed before,
in a VP-Tree we search the interior child of a node $n$ if our target ball intersects the ball $(n.self, n.radius)$,
and we search the exterior child of a node $n$ if our target ball is at all outside the ball $(n.self, n.radius)$.
We formalize this with the below method:

\begin{align*}
query(v : VP-Tree X, target : X, cutoff : \Rnn) \to 2^X = \\ \begin{cases}
    \emptyset & v = \null \\ 
    \bigcup \left.\begin{cases}
        query(v.interior) & d_X(v.self, target) \le v.radius + cutoff \\ 
        query(v.exterior) & d_X(v.self, target) > v.radius - cutoff \\ 
        \{v.self\} & d_X(v.self, target) < cutoff
    \end{cases}\right\} & \otw 
\end{cases}
\end{align*}

Now, in order to use this data structure we need to first construct it.
To do this we follow a similar to path to that of quick sort.
We will choose a random point, and we will find the distance from it to all other points.
We will then partition the points by those closer to median and further than the median.
Those closer than the median will form the interior and those further than the median will form the exterior of a VP-Tree.
We then create a VP-Tree node by setting its self to the random point, interior and exterior as above, 
and radius as distance to the median. 
We will formalize this with the below function.
As a ease of notation, let $\mradius{S}$ simpliy be an approximate the distance from the random point to the 
median of $S$ found with arbitrary percision in linear time\footnote{this can be done with median of medians approach}.

\begin{align*}
index(\{random\} \cup rest: 2^X) \to VP-Tree X = \\ 
\text{VPTree}\{interior&:index(\{x \in rest : d_X(x, random) < \mradius{S}\}), \\
exterior&:index(\{x \in rest : d_X(x, random) \ge \mradius{S}, \}),\\ 
self&:random,\\
radius&:\mradius{S}\}
\end{align*}