To test both Type2Metric and Axe we use the Go standard library.
We parsed most of the functions, type aliases, struct definitons, constants, and methods.
Though some modules such as parts of \texttt{net}, the network module,
dropped to either assembly or foregin calls.
These packages were eladed because they contained functions for which we could not gather types.
We also eladed all testing libraries as they are not used by users of Go.
We then used the go compiler to extract various fragments of code,
and from the compiler data we extract the types of all of these fragments.
We then encoded these types into an Axe specific representation,
with the intention of extensiblity to other languages.
We also gathered any comment block associated with methods, structures, and any of their fields.
This took relatively little time (less than ten seconds) to collect.
In total we gathered 130 packages, 
and each of these packages contained on aveage 94 fragments.
For each package we ran a series of random queries with two cutoffs.
We recorded the results in \ref{dist1plot,dist6plot}.
We also ran queries over all the packages with several different cutoffs
and we plotted the result in \ref{allplot}.

\subsubsection{Experimental Setup}
Our performance 
experiments all revolved around querying data and checking performance gains
by using a VPTree.
We queried for objects similiar to members of the tree at different ratios.
After indexing, we followed the following testing procedure
\begin{itemize}
    \item 
    For each packacge we randomly chose 10 fragments in those packages 
    using clock time to provide randomness.
    \item 
    We then iterated through various cutoffs, for individual packages we 
    used $1, 100, 1000$. 
    For the set of all packages we also tested $0,2,3,5$.
    \item 
    For a each cutoff we ran each query 10 times over 10 queries 
    and then found the averaged the total time for lookup.
    We reported this as time for any given benchmark.
\end{itemize}
For all packages not containing foreign code,
this querying and indexing worked.