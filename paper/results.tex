One question we have is does search even work?
That is, if we query do we actually get reasonable results?
There are not yet great ways to evaluate search appropriateness
Especially when evaluating type search, a new feature to Go.
That said, we present the following argument:
a query gives a good result.
So, a query that we ran was for all types matching.
$\bfloat$ represents a 64-bit floating point and $??$
a hole that matches all types.
This query seems like a reasonable query if we consider a Developer who wants 
to possible raise some number to a power
$$(\bfloat, ??) \to \bfloat$$

We get the following results:
\begin{itemize}
    \item $\textbf{math/Max} : (\bfloat, \bfloat) \to \bfloat$ takes the max of two numbers  \\
    \item $\textbf{math/Ldexp} : (\bfloat, \mathbb{Z} \to \bfloat)$ multiples the first argument by two the power of the second   \\
    \item $\textbf{math/Dim} : (\bfloat, \bfloat) \to \bfloat$ takes the max of the difference of the two numbers and zero  \\
    \item $\textbf{math/Mod} : (\bfloat, \bfloat) \to \bfloat$ takes the floating-point remainder of $x/y$ \\
    \item $\textbf{math/Min} : (\bfloat, \bfloat) \to \bfloat$ takes the min of two numbers  \\
    \item $\textbf{math/Nextafter} : (\bfloat, \bfloat) \to \bfloat$ returns the next floating point after the first argument in the direction of the second. \\
    \item $\textbf{math/Pow} : (\bfloat, \bfloat) \to \bfloat$ returns the first argument raised to the second  \\
    \item $\textbf{math/Remainder} : (\bfloat, \bfloat) \to \bfloat$ returns the IEEE 754 floating-point remainder of x/y \\
    \item $\textbf{math/Hypot} : (\bfloat, \bfloat) \to \bfloat$ pythagorean theorem solver \\
    \item $\textbf{math/remainder} : (\bfloat, \bfloat) \to \bfloat$ internal remainder function \\
\end{itemize}

Looking at the results, these seem pretty good.
However, in future work we'd like to develop a better way of gaging good results