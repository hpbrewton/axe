\documentclass{article}
\usepackage[margin=1in]{geometry}
\usepackage{amsmath}
\usepackage{amssymb}
\usepackage{amsthm}
\usepackage{mathpartir}

\usepackage{cite}

\newcommand{\Rnn}{\mathbb{R}^{\ge 0}}
\newcommand{\otw}{\text{otherwise.}}

\newcommand{\lev}[1]{\text{Lev}_{#1}}
\newcommand{\tlist}[1]{\text{List}\quad #1}
\newcommand{\tail}[1]{#1[1:]}
\newcommand{\head}[1]{#1[0]}

\newtheorem{definition}{Definition}

\begin{document}
{\centering \Huge \textbf{A Code Search Engine for Go} \\ 
\vspace{0.5cm}
\Large
Harrison Brewton \\
Advised by Aws Albarghouthi \\ 
April, 2020 \\
\normalsize
\vspace{0.1cm} \textit{``All programming language proofs are by induction''} \vspace{0.1cm} \par
}
%abstract 
\section{Background}
\section{Overview}
\section{Type2Metric}
\subsection{Levenshtein Combinator}
Suppose we have two lists $[1, 2, 1, 1]$ and $[1, 3, 2]$.
We want to find the distance between these two lists. 
This is a somewhat ill defined question,
but we can divine a reasonable approach to this question.
We can ask what is the minimum cost to change the first list into the second list.
Cost, for this example, can be defined as the fewest insertions, substitutions, and removals required to convert between the two lists.
There are two things to note here, first different substitutions are more expensive than others,
for example $1000 \mapsto 1$ is definitely more expensive than $2 \mapsto 1$.
For simplicity, we will say a substitution is as expensive as the difference between the two numbers are.
With this in mind we can set a substitution or removal to have some finite cost. Why not $3$?
We choose this number somewhat flippantly, but there is a bit of an art to it.
It's worth noting that this constant sets an upper limit to the substitution cost.
This is because any substitution can be achieved with an insertion and a deletion.
So, in this particular example the substitution between any two numbers is really the minimum of their difference and $6$ (the cost of a deletion followed by an insertion).

The Levenshtein edit distance is a commonly used metric for strings.
It gives the number of substitutions, deletions, or additions to edit one string into another string \cite{wiki:levenshtein}.
It is well known that the normal Levenshtein distance on strings satisfies the properties of a metric.
There are a couple of assumptions that the Levenshtein metric makes:
that all substitutions are of the same cost,
that deletions and additions are of the same cost and the same cost as a substitution.
While this is a safe assumption for strings of characters,
it is not general, as there might be better substitutions than others.
For example, 
it is clear that in a real sense $[1, 2, 3]$ is closer to $[2, 2, 3]$ than it is to $[10, 2, 3]$,
as a $1$ is closer to a $2$ than a $10$;
however, the Levenshtein edit distance would be the same from the first string to the second string (1).
To that end we define the Levenshtein Combinator.

\begin{definition}[Levenshtein Combinator]
Assume that $d : (T \times T) \to \Rnn$ is a metric.
Suppose further that $\kappa$ is any positive real.
Then we say the Levenshtein Combinator of $d$ is  $\lev{d} : (\tlist{T} \times \tlist{T}) \to \Rnn$.
Such that it is defined on lists of $T$.
We define the combinator as: 
$$ \lev{d}(a, b) = \begin{cases} 
    \kappa | \tail{a} | & |b| = 0 \\
    \kappa | \tail{b} | & |a| = 0 \\ 
    \text{min} \begin{cases}
        \lev{d}(\tail{a}, b) + \kappa \\
        \lev{d}(a, \tail{b}) + \kappa \\
        \lev{d}(\tail{a}, \tail{b}) + d(\head{a}, \head{b})  % \lev{d}(\init{a}, b) + \kappa
    \end{cases} & \otw
\end{cases}$$
\end{definition}

The Levenshtein Combinator runs almost the same as the traditional Levenshtein edit distance.
The only differences are the generalizations of the $\kappa$ from 1,
and the addition of the metric between elements of the list.
In fact, the original Levenshtein distance can be recovered by substituting $1$ for $\kappa$,
and the characteristic function $1_{a \neq b}$ as the $d$ metric.
We will now show that the Levenshtein combinator of a metric is itself a metric. 

Identity of indiscernibles and symmetry are pretty obvious, and so are omitted.
The triangle inequality also follows in a similar manner as the proof of traditional Levenshtein \cite{j2kun_2014}.
We first make a simple optimality argument of the Levenshtein distance,
arguing that if there were a shorter distance between two strings it would have been found.
Suppose we are considering a transformation between lists $X$ and $Z$,
that is a series of insertion, deletions, and substitutions.
A transformation from $X$ to some $Y$ followed by a transformation from $X$ to some $Z$
is a transformation from $X$ to $Z$.
As the transformation from $X$ to $Z$ is the minimal transformation,
it follows $d(X, Z) \le d(X, Y) + d(Y, Z)$.
For more detail on this proof see \cite{wiki:levenshtein}.
\subsection{Linear Combinator}
We may want to use multiple definitions of distance between some object.
A simple example is finding the taxi cab distance between two points.
Which is the sum of the distances between two points along any finite number of axes.
We may also want to weigh different distances different values.
This is reasonable if we consider the taxi cab analogy where blcoks might be longer north-south 
than they are east-west.
We can apply this idea generally to any product type,
where we want to define the metric of that type as a linear combination of the metrics of each of the fields.

We will now briefly show that the linear combination of metrics is a metric.
Suppose we have a possible metric $d(a, b) = \sum c_id_i(a, b)$, 
where $c_i$ is a positive real number, and $d_i$ is a metric.
Symmetry follows clearly.
Identity of indescernables follows as each term goes to zero, leaving the sum at zero.
The triangle ineqaulity follows $d(a, b) = \sum c_id_i(a, b) \le \sum c_id_i(a, c)+c_id_i(a, b) = d(a, c) + d(b, c)$.
Thus, we can safely take a linear combination of metrics and arrive still at a metric.
\subsection{From Types To Metrics}
\newcommand{\const}{\text{const }}
\newcommand{\sortp}[1]{\text{sortedPairs}(x)}

In Figure \ref{rules} we provide a formalization of rules for generating a metric from a type.
They are largely the same as discussed in the previous subsections, but they have two additions which are worht noting.
First, we haved added maps, a common language type. 
Largely, these are treated like lists;
however to do so we first have to convert them to lists of sorted pairs of keys and values and then we sort them by their key element.
At this point we simply plug them into the levenshtein distance.
Second, we have added pointers.
These just behave as wrappers of their types.
While we are able to prevent some types of pointer cycles by testing if pointers are equal,
this will not terminate for all objects such as two linked lists with one element missing.
More work needs to be done to resolve this.

There is a cheap proof that needs to be made where we show that all of the distance fucntions created here are in fact metrics.
As a base case, it is clear that numbers are metrics.
We have sufficiently shown that all other combinations are metrics in the previous.
So, by induction these are all metrics.
We have implemented these rules in a tool called type to metric which we will discuss later.

\begin{figure}
\begin{mathpar}
\inferrule*[Right=Lists]{\Gamma \vdash x, y : \tlist{T}}
    {d(x, y) = \lev{T}(x, y)} 
\\
\inferrule*[Right=Products]{\Gamma \vdash x, y :  (T_1, T_2, \ldots, T_n) \\ \Gamma \vdash c_i = \const((T_1, T_2, \ldots, T_n))}
    {d(x, y) = \sum_{i=0}^n c_id_{T_i}(x_i, y_i)}
\\
\inferrule*[Right=Sums]{\Gamma \vdash x, y : T_1 | T_2 | \ldots | T_n \\ 
    \Gamma \vdash \kappa = \const((T_1 | T_2 | \ldots | T_n))}
    {d(x, y) = 1_{\text{type}(x) = \text{type}(y)} \min(\kappa, d_{\text{type}(x)}(x, y)) + 
    {1_{\text{type}(x) \neq \text{type}(y)}\kappa}}
\\
\inferrule*[Right=Map]{\Gamma \vdash x, y : \tmap{T_1}{T_2}}{d(x, y) = \lev{T}(\sortp{x}, \sortp{y})}
\\ 
\inferrule*[Right=SameAddressPointer]{\Gamma \vdash x, y : \pointer{T}, x = y}
{d(x, y) = 0}
\\ 
\inferrule*[Right=DifferentAddressPointer]{\Gamma \vdash x, y : \pointer{T}, x \neq y}
{d(x, y) = d_T(\pointer{x}, \pointer{y})}
\\ 
\inferrule*[Right=Number]{\Gamma \vdash x, y: T \\ \Gamma T <: \text{Number}}{d(x, y) = |y-x|}
\end{mathpar}
\caption{Rules for converting types to metrics}
\label{rules}
\end{figure}
\section{Example: Go Code Look Up}
\subsection{Generics}
\section{Metric Trees}
\section{Implementation}
\section{Related Work}
\section{Future Work}
\subsection{Best Approximation}
\section{Conclusion}

\bibliographystyle{unsrt}
\bibliography{biblio}
\end{document}