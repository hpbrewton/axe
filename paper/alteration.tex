Sum types appear rather often.
As a simple example, consider an option type where a value might be present and have some value 
or might not be present and thus have no value.
How might we define a metric for two objects of this type.
Well if both of the objects have values present, then we can simply use the metric for their values (assuming they exist).
If niether exist, then the two objects are exactly the same, and so should have zero distance.
However, what do we do if one exists and the other does not.
One option, and the one taken here, is to simply give them some fixed value.
This choice is clean, but does pin us in a bit.
Consider what happens if the distance between two present values is greater than twice this constant.
We would have broken our metric definition by allowing a shorter route from the first object through an objects of a different type in the alteration
back to the second than the given route between the two objects.
So we revise earlier, and say that the distance between objects of the same type in the alternation is the min of their types metric result 
and the change type constant. 
We formalize this below.

Suppose that we have a tuple $(n, v)$, where $n$ is some natural number less than some $N$,
and a mapping $f : n \to (v \times v \to \R)$.
Then we can define the alteration combinator on these tuples.

\begin{definition}[Alteration Combinator]
Let $T = E \times V$, such that $E$ is some finite set of objects, and $V$ is any other set of objects.
Let $f : E \to (V \times V \to \R)$ be a relation, such that for all $e$ in $E$, 
we have $(f e)$ restricted to $\{ (d, v) \in T : d = e \}$ is a metric on that restriction.
Furthermore, let $\kappa \in \R^{+}$.
Then we define the alteration combinator $Alt : f \to T \times T \to \Rnn$
$$ Alt(f, ((e_1, v_1), (e_2, v_2))) = \begin{cases}
    \min( [f e_1](v_1, v_2), \kappa) & e_1 = e_2 \\ 
    \kappa & \otw
\end{cases}$$
\end{definition}

Symmetry, and identity of indiscernables is pretty straightforward.
The triangle inequality for $Alt(f)$, is a little trickier.
We consider four cases. 
Let $A$, $B$, and $C$ all be in $E$.
We can consider the triangle inequality as triple of three of these types,
and ask where it will hold. 
If all three are of the same type $(A, A, A)$, then the inequality is inherited by that of $(f A)$.
Consider $(A, A, B)$, then the inequality can be seen $(f A)(A, A) \le \kappa \le (f A)(A, A) + \kappa \le (f A)(A, A) + Alt(f)(A, B)$
Consider $(A, B, B)$, then the inequality can be seen $(f A)(A, B) = \kappa \le \kappa + (f B)(B, B)$.
Consider $(A, B, C)$, then the inequality obviously holds.
