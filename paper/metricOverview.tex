Metrics provide a sensible definition of distance:
the distance from an object to itself is none,
the distance between and object and another is same in either direction,
and the distance between an object and another is the shortest way to get there.
Metrics can be formalized simply:

\begin{definition}[Metric]
Let $(X, d)$ space, such that $d: X \times X \to \Rnn$. 
Then $d$ forms a metric over $X$ if three proprties hold:
symmetry or $d(a, b) = d(b, a)$, identity of indescernables or $d(a, b) = 0 \iff a = b$,
and sub-additivity $d(a, b) \le d(a, c) + d(c, b)$ for all $c$.
\end{definition}

Because of this, they are often used for searching for the nearest objects to some query.
They've been used for image processing, computational biology, computer vision, mealody search, amongst others \cite{chen}.
While useful metrics for particular problem spaces can be difficult to create especially when dealing with complicated data types commonly found
in modern databases.
The first goal of this paper is to provide an automatic mechanism which allows a user to provide a type definition,
and from this definition produce a metric which allow the user to compute similarity.
This metric can then be used for further applications such as threshold search, repair, and possibly synthesis.
We implement this in a tool called Type2Metric.