Suppose we have two lists $[1, 2, 1, 1]$ and $[1, 3, 2]$.
We want to find the distance between these two lists. 
This is a somewhat ill defined question,
but we can divine a reasonable approach to this question.
We can ask what is the minimum cost to change the first list into the second list.
Cost, for this example, can be defined as the fewest insertions, substitutions, and removals required to convert between the two lists.
There are two things to note here, first different substitutions are more expensive than others,
for example $1000 \mapsto 1$ is definitely more expensive than $2 \mapsto 1$.
For simplicity, we will say a substitution is as expensive as the difference between the two numbers are.
With this in mind we can set a substitution or removal to have some finite cost. Why not $3$?
We choose this number somewhat flippantly, but there is a bit of an art to it.
It's worth noting that this constant sets an upper limit to the substitution cost.
This is because any substitution can be achieved with an insertion and a deletion.
So, in this particular example the substitution between any two numbers is really the minimum of their difference and $6$ (the cost of a deletion followed by an insertion).

The Levenshtein edit distance is a commonly used metric for strings.
It gives the number of substitutions, deletions, or additions to edit one string into another string \cite{wiki:levenshtein}.
It is well known that the normal Levenshtein distance on strings satisfies the properties of a metric.
There are a couple of assumptions that the Levenshtein metric makes:
that all substitutions are of the same cost,
that deletions and additions are of the same cost and the same cost as a substitution.
While this is a safe assumption for strings of characters,
it is not general, as there might be better substitutions than others.
For example, 
it is clear that in a real sense $[1, 2, 3]$ is closer to $[2, 2, 3]$ than it is to $[10, 2, 3]$,
as a $1$ is closer to a $2$ than a $10$;
however, the Levenshtein edit distance would be the same from the first string to the second string (1).
To that end we define the Levenshtein Combinator.

\begin{definition}[Levenshtein Combinator]
Assume that $d : (T \times T) \to \Rnn$ is a metric.
Suppose further that $\kappa$ is any positive real.
Then we say the Levenshtein Combinator of $d$ is  $\lev{d} : (\tlist{T} \times \tlist{T}) \to \Rnn$.
Such that it is defined on lists of $T$.
We define the combinator as: 
$$ \lev{d}(a, b) = \begin{cases} 
    \kappa | \tail{a} | & |b| = 0 \\
    \kappa | \tail{b} | & |a| = 0 \\ 
    \text{min} \begin{cases}
        \lev{d}(\tail{a}, b) + \kappa \\
        \lev{d}(a, \tail{b}) + \kappa \\
        \lev{d}(\tail{a}, \tail{b}) + d(\head{a}, \head{b})  % \lev{d}(\init{a}, b) + \kappa
    \end{cases} & \otw
\end{cases}$$
\end{definition}

The Levenshtein Combinator runs almost the same as the traditional Levenshtein edit distance.
The only differences are the generalizations of the $\kappa$ from 1,
and the addition of the metric between elements of the list.
In fact, the original Levenshtein distance can be recovered by substituting $1$ for $\kappa$,
and the characteristic function $1_{a \neq b}$ as the $d$ metric.
We will now show that the Levenshtein combinator of a metric is itself a metric. 

Identity of indiscernibles and symmetry are pretty obvious, and so are omitted.
The triangle inequality also follows in a similar manner as the proof of traditional Levenshtein \cite{j2kun_2014}.
We first make a simple optimality argument of the Levenshtein distance,
arguing that if there were a shorter distance between two strings it would have been found.
Suppose we are considering a transformation between lists $X$ and $Z$,
that is a series of insertion, deletions, and substitutions.
A transformation from $X$ to some $Y$ followed by a transformation from $X$ to some $Z$
is a transformation from $X$ to $Z$.
As the transformation from $X$ to $Z$ is the minimal transformation,
it follows $d(X, Z) \le d(X, Y) + d(Y, Z)$.
For more detail on this proof see \cite{wiki:levenshtein}.