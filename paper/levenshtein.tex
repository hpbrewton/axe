The Levenshtein edit distance is a commonly used metric for strings.
It gives the number of substitutions, deletions, or additions to edit one string into another string \cite{wiki:levenshtein}.
It is well known that the normal Levenshtein distance on strings satifies the properties of a metric.
There are a couple of assumptions that the Levenshtein metric makes:
that all substitutions are of the same cost,
that deletions and additions are of the same cost and the same cost as a substitution.
While this is a safe assumption for strings of characters,
it is not genereral, as there might be better substitutions than others.
For example, 
it is clear that in a real sense $[1, 2, 3]$ is closer to $[2, 2, 3]$ than it is to $[10, 2, 3]$,
as a $1$ is closer to a $2$ than a $10$;
however, the Levenshtein edit distance would be the same from the first string to the second string (1).
To that end we define the Levenshtein Combinator.

\begin{definition}[Levenshtein Combinator]
Assume that $d : (T \times T) \to \Rnn$ is a metric.
Suppose further that $\kappa$ is any positive real.
Then we say the Levenshtein Combinator of $d$ is  $\lev{d} : (\tlist{T} \times \tlist{T}) \to \Rnn$.
Such that it is defined on lists of $T$.
We define the combinator as: 
$$ \lev{d}(a, b) = \begin{cases} 
    \kappa | \tail{a} | & |b| = 0 \\
    \kappa | \tail{b} | & |a| = 0 \\ 
    \text{min} \begin{cases}
        \lev{d}(\tail{a}, b) + \kappa \\
        \lev{d}(a, \tail{b}) + \kappa \\
        \lev{d}(\tail{a}, \tail{b}) + d(\head{a}, \head{b})  % \lev{d}(\init{a}, b) + \kappa
    \end{cases} & \otw
\end{cases}$$
\end{definition}

The Levenshtein Combinator runs almost the same as the traditional Levenshtein edit distance.
The only differences are the generalizations of the $\kappa$ from 1,
and the addition of the metric between elements of the list.
In fact, the original Levenshtein distance can be recovred by substituting $1$ for $\kappa$,
and the characteristic function $1_{a \neq b}$ as the $d$ metric.
We will now show that the Levenshtein combinator of a metric is itself a metric. 

Identity of indiscernibles and symmetry are pretty obvious, and so are omitted.
The triangle inequality also follows in a similar manner as the proof of traditional levenshtein \cite{j2kun_2014}.
Suppose we are considering a transformation between lists $X$ and $Z$,
that is a series of insertion, deletions, and substitutions.
A transformation from $X$ to some $Y$ followed by a transformation from $X$ to some $Z$
is a transformation from $X$ to $Z$.
As the transformation from $X$ to $Z$ is the minimal transformation,
it follows $d(X, Z) \le d(X, Y) + d(Y, Z)$.